\documentclass[12pt,letterpaper,titlepage,draft]{article}
\usepackage{amsmath}
\usepackage{amsfonts}
\usepackage{amssymb}
\usepackage[left=1in,right=1in,top=1in,bottom=1in]{geometry}
\usepackage{graphicx}
\usepackage{hyperref}
\usepackage[utf8]{inputenc}

\renewcommand{\refname}{\section{References}}

\title{Thesis Proposal}
\author{George Lesica\\
Department of Computer Science\\
University of Montana\\
Missoula, MT}

\begin{document}
\maketitle

\section{Introduction}

\paragraph{}
The theory of neutral evolution was first proposed by Kimura~\cite{Kimura1984}
to explain observed differences between evolution in populations of larger
organisms and evolution at the molecular level. A population undergoing
neutral evolution is able to ``explore'' genotype space through neutral
mutations. Neutral mutation refers to mutations that have little or no affect
on the fitness of an organism. In biology, this is the result of a many-to-one
relationship between genotypes and phenotypes~\cite{Newman1998}.

\paragraph{}
A population undergoing neutral evolution is likely to exhibit high genotypic
diversity. This presents opportunities for the population. Its members are able
to drift through genotypic space since selective pressure does not operate in
the context of neutral mutations. As a result, there is a greater likelihood
that some population member will ``discover'' a genotype that can easily mutate
into a genotype that maps to a higher-fitness phenotype, elevating the entire
population assuming the mutation fixes.

\subsection{Computational Modeling}

\paragraph{}
In computational modeling we can think of two genotypes that produce the same
phenotype (or which have equal fitness), each of which may be produced by
mutating the other in some small way, as two nodes in a neutral network. Many
such networks may exist for a particular genome, each corresponding to a
particular phenotype or fitness level. In some cases, each of these networks
will be small. In other cases, such networks may be expansive and may permeate
the genotype space. In the case where the neutral network for a given phenotype
(or fitness value) forms a single giant component (or comes close) it is said
to be a percolating neutral network.

\subsubsection{Fitness Landscapes}

\paragraph{}
A set of genotypes and their corresponding phenotypes or fitness levels make up
a fitness landscape, which is simply a useful visual metaphor. The intuition is
that higher ground within the landscape corresponds to higher fitness. Fitness
landscapes can be smooth or rugged. Smooth landscapes have lower peaks, in
other words there is less difference between fit and unfit genotypes.

\paragraph{}
Fitness landscapes can also be correlated or uncorrelated. A correlated
landscape exhibits auto-correlation. This is most naturally understood as the
expected correlation between successive fitness values during a random walk
across the landscape. Visually, a correlated landscape might look similar to
rolling hills, whereas an uncorrelated landscape would appear extremely jagged,
sometimes going from very low to very high in a short distance.

\subsubsection{Adaptive Walks}

\paragraph{}
An adaptive walk over a fitness landscape is the process of choosing a
genotype, then mutating it into a new, fitter genotype and repeating the
process. The fact that the new genotype is always fitter (or possibly at least
as-fit) as the current genotype clearly differentiates an adaptive walk from a
random walk, which can also be useful to the study of fitness landscapes. It is
helpful to know that this process is often called ``hill climbing'', which
matches up with the landscape metaphor quite neatly.

\paragraph{}
There are several different methods for choosing the next genotype to occupy
during an adaptive walk. Some of the common ones are nicely summarized
in~\cite{Nowak2015}. These include fitness-dependent, where the choice is made
randomly, but fitter genotypes are given preference. Alternately, the choice
can be purely random. A greedy strategy chooses the fittest candidate, and
finally, a reluctant strategy chooses the least fit candidate that is fitter
than the current genotype.

\paragraph{}
Fitness landscapes have traditionally been seen as rugged, with many, more or
less independent, peaks and valleys pervading the space. Under these
conditions, it is possible, even likely, that a population undergoing an
adaptive walk will become ``trapped'' at a local optimum. Imagine attempting to
reach the summit of Mount Everest by starting where you are and continuously
walking toward a higher elevation. You will almost certainly fail to reach your
goal.

\subsubsection{Neutral Networks}

\paragraph{}
Adaptive walks on fitness landscapes that possess neutral networks, however,
are less likely to dead-end.

\paragraph{}
We can think of exploring a fitness landscape by following neutral pathways
that may or may not form a percolating (or nearly so) network. Random mutation
drives this process. Eventually (or perhaps frequently), some individual will
mutate off of the neutral network and onto a neutral network with higher
fitness. If this mutation fixes in the population, then the population will
have adapted or evolved.

\paragraph{}
Neutral landscapes, on the other hand, can be more accurately imagined as
mountain ranges, with ridges running along slopes joining together peaks,
saddles between the peaks themselves, and relatively flat valleys and meadows.
Adaptive walks on such landscapes are likely to result in higher fitness or,
put another way, populations are less likely to become stuck at local
optima~\cite{Newman1998}.

\section{Hypotheses}

\subsection{Mutation-Selection Balance}

\paragraph{}
Gavrilets demonstrated~\cite{Gavrilets1997} that, on an uncorrelated landscape,
the percolation threshold for a fitness landscape is inversely correlated with
the number of dimensions under consideration. We hypothesize that this is true
for correlated landscapes as well, given an appropriate balance between
selection and mutation pressures.

\section{Methods}

\paragraph{}
To facilitate testing our hypotheses, we intend to create two pieces of
reusable software, both of which will be released as open source software.

\subsection{NK.jl Library}

\paragraph{}
We have opted to generalize our NK landscape
implementation\footnote{\url{https://github.com/glesica/NK.jl}} so that it may
be used by others in the future. We will provide it as a separate package under
a permissive license that will hopefully allow future work and improvement.  We
will use this library for our own experiments in addition to contributing it to
the community as open source software.

\paragraph{}
We intend to implement several NK variants including NKq and NKp neutral
landscapes. We will also provide implementations of several common techniques
such as adaptive walks and proportional selection from a population.

\subsection{CGP.jl Library}

\paragraph{}
Cartesian Genetic Programming (CGP) eschews the tree structures more commonly
associated with genetic programming in favor of a grid (hence ``Cartesian'') of
functions that can take as their inputs the outputs from any preceding
functions. Our CGP library\footnote{\url{https://github.com/glesica/CGP.jl}}
attempts to combine computation performance with ease-of-use.

\section{Experiments}

\subsection{Shifting Balance Experiment}

\paragraph{}
The first project will test the shifting balance theory proposed by
Wright~\cite{Wright1982}\cite{Wright1931}. We will enumerate the possible
genotypes for small NK and NKp landscapes, then compute the
prominence\footnote{Prominence is a concept from topography. The prominence of
a peak is defined to be the vertical distance between the peak and the
lowest contour line encircling it that has no higher summit.} and identify the
key col\footnote{The key col, also called the saddle point, of a peak is the
lowest point on the highest ridge between a peak and a higher peak that
defines its prominence.} for each fitness peak.

\paragraph{}
Ordinarily, a large population would be unlikely to move between fitness peaks,
even on a neutral landscape. Wright's theory attempted to resolve the apparent
conflict between theory and observation by suggesting that a subpopulation may
be able to sustain temporarily reduced fitness given a sufficiently high rate
of mutation and a sufficiently low rate of gene flow. Wright suggested that a
subpopulations might ``jump'' to a neighboring domain of attraction and climb
the corresponding peak. The beneficial mutation might then fix in the larger
population. We propose that this process is simpler and more likely to occur if
there exist neutral or nearly-neutral ridges between peaks.

\section{Analysis}

\bibliographystyle{acm}
\bibliography{fitness}{}
\end{document}
