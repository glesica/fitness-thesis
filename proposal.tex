\documentclass[12pt,letterpaper,titlepage,draft]{article}
\usepackage[utf8]{inputenc}
\usepackage{amsmath}
\usepackage{amsfonts}
\usepackage{amssymb}
\usepackage{graphicx}
\usepackage[left=1in,right=1in,top=1in,bottom=1in]{geometry}
\title{Thesis Proposal}
\author{George Lesica\\
Department of Computer Science\\
University of Montana\\
Missoula, MT}
\begin{document}
\maketitle

\section*{Introduction}

\paragraph{}
The theory of neutral evolution was first proposed by Kimura~\cite{Kimura1984}
to explain observed differences between evolution in populations of larger
organisms and evolution at the molecular level. In neutral evolution a
population diffuses through genotype space along neutral pathways that form a
percolating network. That is, it is possible to ``move'' from one genotype to
the next without a difference in fitness. Random mutation drives this process.
Eventually, some individual will mutate off of the neutral network and onto a
neutral network with higher fitness, and this mutation will fix within the
population with some probability.

\paragraph{}
Neutrality may be of interest to a variety of fields outside the biological
sciences, including evolutionary computation. Fitness landscapes have
traditionally been seen as rugged, with many, more or less independent, peaks
and valleys pervading the space. Under these conditions, it is possible, even
likely, that a population undergoing an adaptive walk on such a landscape will
become ``trapped'' at a local optimum.

\paragraph{}
Neutral landscapes, on the other hand, can be seen more as mountain ranges,
with ridges running along slopes joining together peaks, and saddles between
the peaks themselves. Adaptive walks on such landscapes are likely to result in
higher fitness or, put another way, populations are less likely to become stuck
at local optima~\cite{Newman1998}.

\section*{Goals}

\section*{Projects}

\paragraph{}
The first project will test the shifting balance theory proposed by
Wright~\cite{Wright1982}. We will enumerate the possible genotypes for small NK
and NKp landscapes, then compute the prominence\footnote{Prominence is a
concept from topography. The prominence of a peak is defined to be the
vertical distance between the peak and the lowest countour line encircling it
that has no higher summit.} and identify the key col\footnote{The key col, also
called the saddle point, of a peak is the lowest point on the highest ridge
between a peak and a higher peak that defines its prominence.} for each fitness
peak.

\paragraph{}
Ordinarily, a large population would be unlikely to move between fitness peaks, even on a neutral landscape. Wright's theory attempted to resolve the apparent conflict between theory and observation by suggesting that subpopulations may be able to sustain temporarily reduced fitness given a sufficiently high rate of mutation. Wright suggested that a subpopulations might ``jump'' to a neighboring domain of attraction and climb the corresponding peak. The beneficial mutation might then fix in the larger population. We propose that this process is simpler and more likely to occur if there exist neutral or nearly-neutral ridges between peaks.

\section*{Analysis}

\bibliography{fitness}{}
\bibliographystyle{acm}
\end{document}
