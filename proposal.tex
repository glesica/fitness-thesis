\documentclass[12pt,letterpaper,titlepage,draft]{article}
\usepackage{amsmath}
\usepackage{amsfonts}
\usepackage{amssymb}
\usepackage[left=1in,right=1in,top=1in,bottom=1in]{geometry}
\usepackage{graphicx}
\usepackage{hyperref}
\usepackage[utf8]{inputenc}

\renewcommand{\refname}{\section{References}}

\title{Thesis Proposal:\\
Selection and Mutation on Correlated\\
Neutral Fitness Landscapes}

\author{George Lesica\\
Department of Computer Science\\
University of Montana\\
Missoula, MT\\
\\
Advisor: Alden Wright}

\begin{document}
\maketitle

\section{Introduction}

\subsection{Biological Neutrality}

\paragraph{}
The theory of neutral evolution was first proposed by Kimura~\cite{Kimura1984}
to explain observed differences between evolution in populations of larger
organisms and evolution at the molecular level. His theory made two
predictions, as laid out by Hughes~\cite{Hughes2007}. First, ``that most
polymorphisms are selectively neutral and are maintained by genetic drift''.
Second, ``that most changes at the molecular level that are fixed over
evolutionary time are selectively neutral and are fixed by drift.''

\paragraph{}
Neutral evolution enables a population to ``explore'' sequence space via
neutral mutation. Neutral mutation itself refers to mutations that have little or
no affect on the fitness of an organism, which is not to say that they have no
affect at all. In biology, this is the result of the many-to-one relationship
between genotypes and phenotypes~\cite{Newman1998}.

\paragraph{}
A population undergoing neutral evolution is likely to exhibit high genotypic
diversity. This presents opportunities for the population. Its members are able
to explore the sequence space since selective pressure does not operate in the
context of neutral mutations. As a result, there is a greater likelihood that
some population member will ``discover'' a genotype that maps to a
higher-fitness phenotype, elevating the entire population when, and if, the
mutation reaches fixation.

\subsection{Computational Modeling}

\paragraph{}
In computational modeling we can think of two genotypes that produce the same
phenotype (or which have equal fitness), each of which may be produced by
mutating the other in some small way, as two nodes in a neutral network. Many
such networks may exist for a particular genome, each corresponding to a
particular phenotype or fitness level. In some cases, each of these networks
will be small. In other cases, such networks may be expansive and may permeate
the sequence space. In the case where the neutral network for a given phenotype
(or fitness value) forms a single giant component (or comes close) it is said
to be a percolating neutral network.

\subsubsection{Fitness Landscapes}

\paragraph{}
Evolutionary models are composed of a set of genotypes and their corresponding
phenotypes (or simply fitness values).  If we consider an arbitrary model with
only two dimensions, and take fitness as a third dimension, we arrive at a
useful visual metaphor which resembles a topographical map. This motivates the
term ``fitness landscape''. The intuition is that higher ground within the
landscape corresponds to higher fitness.

\paragraph{}
More rigorously, a fitness landscape can be defined as ``a mapping from a
configuration space that is equipped with some notion of adjacency, nearness,
distance or accessibility, into the real numbers.''~\cite{Calcott2008}

\paragraph{}
One important statistical property of a fitness landscape is its ``correlation
structure'', which can be defined simply as ``how similar the fitness values of
one-mutant neighbors in the space are''~\cite{Kauffman1993}. More specifically,
the correlation of a landscape is the expected autocorrelation of a random walk
(see below for a discussion of random and adaptive walks) over
it~\cite{Weinberger1990}. Fitness landscapes are often referred to as
``correlated'' or ``uncorrelated'', depending on the frame of reference.
Visually, an uncorrelated landscape would appear jagged, with some neighboring
points in the space having radically different fitness values.

\paragraph{}
Fitness landscapes are sometimes also referred to as ``smooth'' or ``rugged''.
While these terms do not have rigorous universal definitions, they are
generally defined as landscapes with relatively high autocorrelation and low
autocorrelation, respectively. In other words, on a smooth landscape, the
fitness at a particular point in sequence space can be expected to carry more
information about the fitnesses of its neighbors~\cite{Kauffman1993}.

\subsubsection{Random Walks}

\paragraph{}
We will define a walk through sequence space (or over a fitness landscape,
depending on your point of view) to be a path consisting of a sequence of
genotypes, each of which differs from its neighbors, the genotypes before and
after it in the sequence, by a single mutation.

\paragraph{}
A random walk is a path that is constructed recursively by starting at a
genotype, choosing a random one-mutant neighbor (the choice may be uniformly
random or use some other distribution or algorithm), then repeating the process
for the neighbor.

\subsubsection{Adaptive Walks}

\paragraph{}
An adaptive walk is similar to a random walk. However, instead of choosing
one-mutant neighbors at random, they are chosen with an eye toward improving
the fitness of the selected genotype. The path corresponding to a particular
adaptive walk, then, has the property that each genotype on the path is fitter
than the one that came before it. This process is often referred to as
``hill-climbing'', which corresponds nicely with the fitness landscape as a
visual metaphor.

\paragraph{}
There are several different methods for choosing the next genotype during an
adaptive walk. Some of the common ones are nicely summarized
in~\cite{Nowak2015}. These include fitness-dependent, where the choice is made
randomly, but fitter genotypes are given preference. Alternately, the choice
can be purely random. A greedy strategy chooses the fittest candidate, and
finally, a reluctant strategy chooses the least fit candidate that is fitter
than the current genotype.

\paragraph{}
On rugged fitness landscapes, adaptive walks can become ``stuck'' at local
optima because of the constraint that each genotype must be fitter than the one
before it. In fact, this outcome is quite likely. Think of trying to climb
Mount Everest by starting at a random point on the surface of the Earth and
only moving uphill. You would almost certainly end up stuck at the top of some
small hill.

\subsubsection{Neutral Networks}

\paragraph{}
Neutral fitness landscapes can be more accurately imagined as
mountain ranges, with ridges running along slopes joining together peaks,
saddles between the peaks themselves, and relatively flat valleys and meadows.
Adaptive walks on such landscapes are likely to result in higher fitness, as
demonstrated by Barnett~\cite{Barnett1998} or, put another way, populations are
less likely to become stuck at local optima~\cite{Newman1998}.

\paragraph{}
Another way to describe a neutral landscape is as a set of neutral basins, each
with some number of passageways that lead to other basins of higher or lower
fitness (a bit like a multi-level waterfall). A balance between selection and
deleterious mutation prevents the population from falling ``back'' into a lower
fitness basin~\cite{Crutchfield1999}.

\paragraph{}
We can think of exploring a fitness landscape by following neutral pathways
that may or may not form a percolating (or nearly so) network. Random mutation
drives this process. Eventually (or perhaps frequently), some individual will
mutate off of the neutral network and onto a neutral network with higher
fitness. If this mutation fixes in the population, then the population will
have adapted or evolved.

\section{NK Fitness Landscapes}

\paragraph{}
The NK fitness landscape, developed by Kauffman~\cite{Kauffman1993}, is a
fitness model that features tunable ruggedness. It derives its name from the
fact that an NK landscape is parameterized by two values: $N$, the number of
loci in each genotype, and $K$, the number of loci that are considered
epistatically linked to each locus.

\paragraph{}
The value of $K$ determines how rugged the landscape will be. Higher values
yield greater ruggedness. The special case where $K=0$ yields a landscape with
just a single peak.

\subsection{Construction and Evaluation}

\paragraph{}
An NK landscape is constructed by first choosing values for $N$ and $K$. Then,
for each locus in $N$, $K$ other loci are chosen at random. For each
combination of alleles for those $K+1$ loci (two alleles, 0 and 1, are
typically used for simplicity), a fitness value between zero and one is chosen.

\paragraph{}
Evaluation is done by computing the average fitness across the contributions
from each locus. The contribution for a single locus is computed using the
correct fitness based on the values of the $K+1$ relevant loci and the fitness
table computed when the landscape was created\footnote{In reality, because the
memory required to store such a table grows exponentially with $K$, fitness
values are more commonly generated on-demand. The result, however, is
unchanged.}.

\subsection{Variants}

\paragraph{}
One noteworthy, but perhaps not entirely obvious, implication of the NK design
is that the resultant landscape is almost certainly not neutral. The following
two variants adapt the general NK landscape to add neutrality.

\subsubsection{NKq Landscapes}

\paragraph{}
The NKq landscape developed by Newman and Engelhardt~\cite{Newman1998} achieves
neutrality using integral fitness values in the tables described above. Instead
of values between zero and one, integers in the interval $\left[0,F-1\right]$
are used, where $F$ is chosen in advance and is often set to 2.

\subsubsection{NKp Landscapes}

\paragraph{}
The NKp landscape, due to Barnett~\cite{Barnett1998}, adds an additional
parameter $p$ which is assigned from the interval $\left[0,1\right]$. When
fitness values are assigned during creation of the landscape, fitness of
exactly 0 is chosen with probability $p$. This means that some loci will have
make no contribution to genotype fitness for certain values of the $K$ loci to
which they are linked. A change at one of these loci will then have no impact
on the fitness of the genotype.

\section{Hypotheses}

\subsection{Mutation-Selection Balance}

\paragraph{}
Gavrilets demonstrated~\cite{Gavrilets1997} that the percolation threshold for
an uncorrelated fitness landscape scales with the inverse of the number of
dimensions under consideration. The percolation threshold is the probability
that a given genotype is part of a particular neutral network (Gavrilets
limited fitness values to ``high'' and ``low'', so the neutral network in
question was the high fitness network).

\paragraph{}
In other words, as the number of dimensions goes up, it takes a smaller share
of the nodes to create a percolating neutral network. This makes sense because
each additional dimension increases the number of ``routes'' between two
points. For example, cross the Grand Canyon is quite an adventure in a car or
on foot, but if a helicopter is available, it becomes a fairly simple task.

\paragraph{}
We hypothesize that this holds for correlated landscapes as well, given an
appropriate balance between selection and mutation pressures.

\section{Methods}

\paragraph{}
To facilitate testing our hypotheses, we intend to create two pieces of
reusable software, both of which will be released as open source software.

\subsection{NK.jl Library}

\paragraph{}
We have opted to generalize our NK landscape
implementation\footnote{\url{https://github.com/glesica/NK.jl}} so that it may
be used by others in the future. We will provide it as a separate package under
a permissive license that will hopefully allow future work and improvement.  We
will use this library for our own experiments in addition to contributing it to
the community as open source software.

\paragraph{}
We intend to implement several NK variants including NKq and NKp neutral
landscapes. We will also provide implementations of several common techniques
such as adaptive walks and proportional selection from a population.

\subsection{CGP.jl Library}

\paragraph{}
Cartesian Genetic Programming (CGP) eschews the tree structures more commonly
associated with genetic programming in favor of a grid (hence ``Cartesian'') of
functions that can take as their inputs the outputs from any preceding
functions. Our CGP library\footnote{\url{https://github.com/glesica/CGP.jl}}
attempts to combine computation performance with ease-of-use.

\section{Experiments}

\subsection{Shifting Balance Experiment}

\paragraph{}
The first project will test the shifting balance theory proposed by
Wright~\cite{Wright1982}\cite{Wright1931}. We will enumerate the possible
genotypes for small NK and NKp landscapes, then compute the
prominence\footnote{Prominence is a concept from topography. The prominence of
a peak is defined to be the vertical distance between the peak and the
lowest contour line encircling it that has no higher summit.} and identify the
key col\footnote{The key col, also called the saddle point, of a peak is the
lowest point on the highest ridge between a peak and a higher peak that
defines its prominence.} for each fitness peak.

\paragraph{}
Ordinarily, a large population would be unlikely to move between fitness peaks,
even on a neutral landscape. Wright's theory attempted to resolve the apparent
conflict between theory and observation by suggesting that a subpopulation may
be able to sustain temporarily reduced fitness given a sufficiently high rate
of mutation and a sufficiently low rate of gene flow. Wright suggested that a
subpopulations might ``jump'' to a neighboring domain of attraction and climb
the corresponding peak. The beneficial mutation might then fix in the larger
population. We propose that this process is simpler and more likely to occur if
there exist neutral or nearly-neutral ridges between peaks.

\section{Analysis}

\bibliographystyle{acm}
\bibliography{fitness}{}
\end{document}
