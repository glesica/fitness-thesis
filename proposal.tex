\documentclass[12pt,letterpaper,titlepage,draft]{article}
\usepackage[utf8]{inputenc}
\usepackage{amsmath}
\usepackage{amsfonts}
\usepackage{amssymb}
\usepackage{graphicx}
\usepackage[left=1in,right=1in,top=1in,bottom=1in]{geometry}
\author{George Lesica}
\title{Thesis Proposal}
\begin{document}
\maketitle

\section{Introduction}

\paragraph{}
The theory of neutral evolution was first proposed by Kimura~\cite{Kimura1984}
to explain observed differences between evolution in populations of larger
organisms and evolution at the molecular level. In neutral evolution a
population diffuses through genotype space along neutral pathways that form a
percolating network. That is, it is possible to ``move'' from one genotype to
the next without a difference in fitness. Random mutation drives this process.
Eventually, some individual will mutate off of the neutral network and onto a
neutral network with higher fitness, and this mutation will fix within the
population with some probability.

\paragraph{}
Neutrality may be of interest to a variety of fields outside the biological
sciences, including evolutionary computation. Fitness landscapes have
traditionally been seen as rugged, with many, more or less independent, peaks
and valleys pervading the space. Under these conditions, it is possible, even
likely, that a population undergoing an adaptive walk on such a landscape will
become ``trapped'' at a local optimum.

\paragraph{}
Neutral landscapes, on the other hand, can be seen more as mountain ranges,
with ridges running along slopes joining together peaks, and saddles between
the peaks themselves. Adaptive walks on such landscapes are likely to result in
higher fitness or, put another way, populations are less likely to become stuck
at local optima~\cite{Newman1998}.

\section{Goals}

\section{Projects}

\section{Analysis}

\bibliography{fitness}{}
\bibliographystyle{acm}
\end{document}
