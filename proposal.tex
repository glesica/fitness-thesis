\documentclass[12pt,letterpaper,titlepage]{article}
\usepackage{amsmath}
\usepackage{amsfonts}
\usepackage{amssymb}
\usepackage{caption}
\usepackage{csquotes}
\usepackage[left=1in,right=1in,top=1in,bottom=1in]{geometry}
\usepackage{graphicx}
\usepackage{hyperref}
\usepackage[utf8]{inputenc}

\renewcommand{\refname}{\section{References}}

\title{Thesis Proposal:\\
Selection and Mutation on Correlated\\
Neutral Fitness Landscapes}

\author{George Lesica\\
Department of Computer Science\\
University of Montana\\
Missoula, MT\\
\\
Advisor: Alden Wright}

\begin{document}
\maketitle

\section{Introduction}

\subsection{Biological Neutrality}

\paragraph{}
Neutral evolution enables a population to ``explore'' sequence space via
neutral mutation. Neutral mutation itself refers to mutations that have little
or no effect on the fitness of an organism.\footnote{This is not to say that a
neutral mutation has no effect on the phenotype of an organism. Just that
whatever effect it does have does not significantly impact the fitness of the
organism.} In biology, this is the result of the many-to-one relationship
between genotypes and phenotypes~\cite{Newman1998}. While neutral mutations
have no significant effect on fitness, they can pave the way, and can be
necessary conditions, for future beneficial mutations. This type of neutral
mutation is known as a potentiating mutation.

\paragraph{}
The theory of neutral evolution was first proposed by Kimura~\cite{Kimura1984}
to explain observed differences between evolution in populations of larger
organisms and evolution at the molecular level. His theory made two
predictions, as laid out by Hughes~\cite{Hughes2007}. First, ``that most
polymorphisms are selectively neutral and are maintained by genetic drift''.
Second, ``that most changes at the molecular level that are fixed over
evolutionary time are selectively neutral and are fixed by drift.''

\paragraph{}
A population undergoing neutral evolution is likely to exhibit high genotypic
diversity~\cite{Huynen1996a}. This presents opportunities for the population.
Its members are able to explore the sequence space since selective pressure
does not operate in the context of neutral mutations. As a result, there is a
greater likelihood that some population member will ``discover'' a genotype
that maps to a higher-fitness phenotype, elevating the entire population when,
and if, the mutation reaches fixation.

\subsection{Computational Modeling}

\paragraph{}
In computational modeling we can think of two genotypes that produce the same
phenotype (or which have equal fitness), each of which may be produced by
mutating the other in some small way, as two nodes in a neutral network. The
neutral network itself is the set of genotypes that map to the given phenotype
or fitness level.

\paragraph{}
Two genotypes within a neutral network are connected if each
can mutate into the other, usually by a single-point mutation (this is
considered to be a biologically-feasible constraint since the probability of a
mutation at any given locus within a gene is extremely low~\cite{Orr2005}).

\paragraph{}
Many such networks may exist for a particular genome, each corresponding to a
particular phenotype or fitness level. In some cases, each of these networks
will be small. In other cases, such networks may be expansive and may permeate
the sequence space.

\paragraph{}
A connected component of a network that contains a constant fraction, as
opposed to $O\left(\log n\right)$ (where $n$ is the number of nodes in the
network), of the nodes in the entire network is known as a giant component. In
the case where the neutral network for a given phenotype (or fitness value)
forms a giant component (or comes close) it is said to be a percolating neutral
network.

\paragraph{}
Put another way, a percolating network is one for which most nodes that are not
in the network are nearby neighbors of one or more nodes that are in the
network. For example, the human circulatory system may be thought of as a
percolating network. Most cells in the body that are not part of the
circulatory system are still within a few cells distance of one or more cells
that are part of the circulatory system.

% TODO: Include diagram from Gavrilets

\subsubsection{Fitness Landscapes}

\paragraph{}
Evolutionary models are composed of a set of genotypes and their corresponding
phenotypes (or simply fitness values).  If we consider an arbitrary model with
only two dimensions, and take fitness as a third dimension, we arrive at a
useful visual metaphor which resembles a topographical map. This motivates the
term ``fitness landscape''. The intuition is that higher ground within the
landscape corresponds to higher fitness~\cite{Wright1932}.

\paragraph{}
More rigorously, a fitness landscape can be defined as ``a mapping from a
configuration space that is equipped with some notion of adjacency, nearness,
distance or accessibility, into the real numbers''~\cite{Calcott2008}.

\paragraph{}
One important statistical property of a fitness landscape is its ``correlation
structure'', which can be defined simply as ``how similar the fitness values of
one-mutant neighbors in the space are''~\cite{Kauffman1993}. More specifically,
the correlation of a landscape is the expected autocorrelation of a random walk
over it~\cite{Weinberger1990} (see below for a discussion of random and
adaptive walks).

\paragraph{}
Fitness landscapes are often referred to as ``correlated'' or ``uncorrelated''.
An uncorrelated landscape is characterized by low correlation between
one-mutant neighboring genotypes. In other words, the fitness of a given
genotype provides little or no information about the fitnesses of its
one-mutant neighbors. Visually, an uncorrelated landscape would appear jagged,
with some neighboring points in the space having radically different fitness
values.

\paragraph{}
Fitness landscapes are sometimes also referred to as ``smooth'' or ``rugged''.
While these terms do not have rigorous universal definitions, they are
generally defined as landscapes with relatively high autocorrelation and low
autocorrelation, respectively. In other words, on a smooth landscape, the
fitness at a particular point in sequence space can be expected to carry more
information about the fitnesses of its neighbors~\cite{Kauffman1993}.

\subsubsection{Walks}

\paragraph{}
We will define a walk through sequence space (or over a fitness landscape,
depending on your point of view) to be a path consisting of a sequence of
genotypes, each of which differs from its neighbors, the genotypes before and
after it in the sequence, by a single mutation.

\paragraph{}
A random walk is a path along which neighbors are chosen randomly (the choices
may be uniformly random or may use some other distribution or algorithm).

\subsubsection{Adaptive Walks}

\paragraph{}
An adaptive walk is similar to a random walk. However, instead of choosing
one-mutant neighbors at random, they are chosen with an eye toward improving
the fitness of the selected genotype. The path corresponding to a particular
adaptive walk, then, has the property that each genotype on the path is fitter
than (or at least as fit as) the one that came before it. This process is often
referred to as ``hill-climbing'', which corresponds nicely with the fitness
landscape as a visual metaphor.

\paragraph{}
There are several different methods for choosing the next genotype during an
adaptive walk. Some of the common ones are nicely summarized
in~\cite{Nowak2015}. These include fitness-dependent, where the choice is made
randomly, but fitter genotypes are given preference. Alternately, the choice
may be purely random (among the fitter genotypes). A greedy strategy chooses
the fittest candidate, and finally, a reluctant strategy chooses the least fit
candidate that is at least as fit as the current genotype.

\paragraph{}
If any of the above strategies are allowed to consider mutants that are equally
fit in addition to those that are fitter, the adaptive walk may explore fitness
plateaus, or sets of genotypes with equal fitness.

\paragraph{}
On rugged fitness landscapes, adaptive walks can become ``stuck'' at local
optima because of the constraint that each genotype must be at least as-fit
than the one before it. In fact, this outcome is quite likely. Think of trying
to climb Mount Everest by starting at a random point on the surface of the
Earth and only moving uphill. You would almost certainly end up stuck at the
top of some small hill.

\paragraph{}
The only exception to this outcome would be if you started within the ``basin
of attraction'' of Mount Everest. That is, the region in which your chosen walk
would actually take you to the top of the mountain. These basins may be quite
small, and their size is not necessarily related to the height of their
corresponding fitness peak.

\subsubsection{Neutral Landscapes}

\paragraph{}
Neutral fitness landscapes are difficult to describe intuitively since many of
their characteristics might also be present in non-neutral landscapes. However,
Crutchfield and van Nimwegen~\cite{Crutchfield1999} give a good intuitive
characterization. They describe neutral landscapes as networks of
interconnected neutral basins, see Figure~\ref{fig:crutchfield-basins}. In a
way, their description recalls the board game ``Chutes and Ladders'', though
the mechanics are slightly different.

\begin{figure}
    \centering
    \includegraphics[width=0.8\textwidth]{crutchfield-basins}
    \caption{Crutchfield's neutral basins.}
\label{fig:crutchfield-basins}
\end{figure}

\paragraph{}
Crutchfield describes neutral fitness landscapes as a set of neutral basins,
each with some number of passageways that lead to other basins of higher or
lower fitness (a bit like a multi-level waterfall). A balance between selection
and deleterious mutation prevents the population from falling ``back'' into a
lower fitness basin~\cite{Crutchfield1999}.

\subsubsection{Evolution on Neutral Landscapes}

\paragraph{}
Adaptive walks that allow fitness-neutral steps on neutral fitness landscapes
are likely to result in higher fitness, as demonstrated by
Barnett~\cite{Barnett1998} or, put another way, populations are less likely to
become stuck at local optima~\cite{Newman1998}.

\paragraph{}
Neutral networks also may exhibit the ``perpetual innovation'' discussed by
Huynen~\cite{Huynen1996}. This means that a population evolving on a neutral
network continues to ``discover'' new phenotypes or (higher) fitness levels at
a rate that is approximately constant over time. Barnett uses a slightly
stronger version of this concept, which he calls ``constant innovation'', by
adding the constraint that the rate of discovery is similar to that of an
unconstrained random walk.

\paragraph{}
We can think of exploring a fitness landscape by following neutral pathways
that may or may not form a percolating (or nearly so) network. Random mutation
drives this process. Eventually (or perhaps frequently), some individual will
mutate off of the neutral network and onto a neutral network with higher
fitness. If this mutation fixes in the population, then the population will
have adapted or evolved.

\subsubsection{Error Thresholds}

% TODO: Add explanation of population genetics probability / time to fixation.
% TODO: Add discussion of Eigen model.

\paragraph{}
In the context of evolution (both computational and biological) an error
threshold can be thought of as a phase transition between order and disorder,
or homogeneity and heterogeneity, either genotypic or phenotypic, as the rate
of mutation changes.  Put another way, the error threshold ``is the error rate
of replication above which the sudden onset of the population delocalization
from the fittest genotype occurs despite Darwinian
selection''~\cite{Takeuchi2007}.

\paragraph{}
There are two error thresholds that are important in population-based
evolution. The genotypic error threshold is the probability of a single-locus
mutation. When the genotypic error threshold is greater than zero and the
population is evolving on a percolating neutral network the population will
spread out through sequence space. In fact, Huynen et al.\ found that after
roughly 500 generations all of the original sequences had disappeared from the
population.

\paragraph{}
At the same time, however, the phenotypic error threshold, the point beyond
which the dominant phenotype will disappear from the population, is higher that
the genotypic error threshold on neutral networks. This means that while the
original sequences had disappeared after 500 generations, the dominant
phenotype survived until the end of the simulation (1,460 generations).

\section{NK Fitness Landscapes}

\paragraph{}
The NK fitness landscape, developed by Kauffman~\cite{Kauffman1993}, is a
fitness model that features tunable ruggedness. It derives its name from the
fact that an NK landscape is parameterized by two values: $N$, the number of
loci in each genotype, and $K$, the number of loci that are considered
epistatically linked to each locus.

\paragraph{}
The value of $K$ determines how rugged the landscape will be. Higher values
yield greater ruggedness. The special case where $K=0$ yields a landscape with
just a single peak.

\subsection{Construction and Evaluation}

\paragraph{}
An NK landscape is constructed by first choosing values for $N$ and $K$. Then,
for each locus out of the $N$ loci, $K$ other loci are chosen at random. For
each combination of alleles for those $K+1$ loci (two alleles, 0 and 1, are
typically used for simplicity), a fitness value between zero and one is chosen.

\paragraph{}
Evaluation is done by computing the average fitness across the contributions
from each locus. The contribution for a single locus is computed using the
correct fitness based on the values of the $K+1$ relevant loci and the fitness
table computed when the landscape was created\footnote{In reality, because the
memory required to store such a table grows exponentially with $K$, fitness
values are more commonly generated on-demand. The result, however, is
unchanged.}.

\subsubsection{Example NK Landscape}

\paragraph{}
We will choose $N=3$, and $K=1$ because they yield a simple landscape that
nonetheless exercises all of the model's features. For each of the $N$ loci, in
other words, for each of the slots in a genotype that is part of this model, we
chose 1 other locus to associate with it (Kauffman refers to them as
epistatically linked). Below is an example set of assignments.

\begin{displaymath}
    \left[3,3,1\right]
\end{displaymath}

\paragraph{}
In this particular case, the first locus is linked to the third locus. The
second locus is linked to the third locus as well. Finally, the third locus is
linked to the first locus. Note that nothing is linked to the second locus,
this is perfectly valid.

\paragraph{}
The next step in constructing an NK model, at least conceptually, is to produce
a fitness table for each locus. In this case, $K$ is small, so we can produce a
complete table for each locus. We will assume two alleles and represent them as
0 and 1.

\begin{table}
    \captionsetup{labelformat=empty}
    \parbox{.3\linewidth}{\centering
        \begin{tabular}{l l}
            $\left[0, 0\right]$ & 0.887 \\
            $\left[0, 1\right]$ & 0.411 \\
            $\left[1, 0\right]$ & 0.799 \\
            $\left[1, 1\right]$ & 0.075 \\
        \end{tabular}
        \caption{Table 1}
    }
    \parbox{.3\linewidth}{\centering
        \begin{tabular}{l l}
            $\left[0, 0\right]$ & 0.757 \\
            $\left[0, 1\right]$ & 0.267 \\
            $\left[1, 0\right]$ & 0.355 \\
            $\left[1, 1\right]$ & 0.909 \\
        \end{tabular}
        \caption{Table 2}
    }
    \parbox{.3\linewidth}{\centering
        \begin{tabular}{l l}
            $\left[0, 0\right]$ & 0.830 \\
            $\left[0, 1\right]$ & 0.162 \\
            $\left[1, 0\right]$ & 0.657 \\
            $\left[1, 1\right]$ & 0.211 \\
        \end{tabular}
        \caption{Table 3}
    }
\end{table}

\paragraph{}
We can now evaluate the fitness of any genotype that is valid for our model
(any string of three bits). To do this, we look up a fitness value for each
locus in its table, then average them (to keep the overall fitness value
between 0 and 1). For example, we will evaluate the string
$\left[1,0,1\right]$. First, we look up the fitness for each locus:

\begin{displaymath}
    \left[0.075, 0.267, 0.211\right].
\end{displaymath}

\paragraph{}
Then we average them to yield the fitness for the genotype as a whole: $0.184$.
Note that we used the fourth entry in the first table, because the first locus
is linked to the third locus, and in our example genotype, they are both equal
to 1, so we used the $\left[1, 1\right]$ entry in the first table.

\subsection{NK Variants}

\paragraph{}
One noteworthy, but perhaps not entirely obvious, implication of the NK design
is that the resultant landscape is almost certainly not neutral, at least
according to a strict definition of the term (see our hypotheses). The
following two variants adapt the general NK landscape to add neutrality.

\subsubsection{NKq Landscapes}

\paragraph{}
The NKq landscape developed by Newman and Engelhardt~\cite{Newman1998} achieve
neutrality using integral fitness values in the tables described above. Instead
of values between zero and one, integers in the interval $\left[0,F-1\right]$
are used, where $F$ is chosen in advance and is often set to 2.

\subsubsection{NKp Landscapes}

\paragraph{}
The NKp landscape, due to Barnett~\cite{Barnett1998}, adds an additional
parameter $p$ which is assigned from the interval $\left[0,1\right]$. When
fitness values are assigned during creation of the landscape, fitness of
exactly 0 is chosen with probability $p$. This means that for some combinations
of alleles, changing one allele does not change the corresponding contribution
to the fitness of the genotype.

\subsection{Spatially Structured Populations}

\paragraph{}
A population is considered to be spatially structured, or subdivided, if it is
partitioned into two or more sub-populations and gene flow between
sub-populations is limited. Population structure favors exploration of a
fitness landscape over exploitation of nearby beneficial mutations.

\paragraph{}
Structured populations may hold an advantage on rugged
landscapes~\cite{Nahum2015} as they can explore distant peaks and are less
likely to become trapped at local optima.

\section{Hypotheses}

\subsection{Spatial Subdivision and Reluctant Walks}

\paragraph{}
We conjecture that the adaptive dynamics of a reluctant adaptive walk are
approximately equivalent to those of a spatially subdivided population
undergoing adaptation for certain rugged landscapes. The logic here is that
both processes sacrifice some exploitation for greater exploration.

\paragraph{}
Since subdivided populations are smaller than the overall population size,
selection should be weaker. In this situation, the population is more likely to
escape a domain of attraction for another, with a potentially higher peak.
Likewise for a reluctant adaptive walk.

\subsection{Starting Position}

\paragraph{}
In~\cite{Nahum2015}, Nahum, et al.\ evolve two populations, one structure, one
unstructure, on the same landscape. Both populations start from the same
starting conditions far from any peak. If the structured population, the
``tortoise'', ends up at a higher fitness peak, the landscape is taken to have
been more rugged, since rugged landscapes confer an advantage to structured
populations.

\paragraph{}
However, this finding is probably sensitive to starting conditions. For
example, if the population, as occurs in biological systems, starts the
evolutionary process on a local fitness peak then a structured population may
be at an even greater advantage since each sub-population will experience less
selection pressure and is therefore more likely to ``escape'' the peak.

\section{Experiments}

\section{Methods}

\paragraph{}
To facilitate testing our hypotheses, we intend to create two pieces of
reusable software, both of which will be released as open source software.

\subsection{NK.jl Library}

\paragraph{}
We have opted to generalize our NK landscape
implementation\footnote{\url{https://github.com/glesica/NK.jl}} so that it may
be used by others in the future. We will provide it as a separate package under
a permissive license that will hopefully allow future work and improvement.  We
will use this library for our own experiments in addition to contributing it to
the community as open source software.

\paragraph{}
We intend to implement several NK variants including NKq and NKp neutral
landscapes. We will also provide implementations of several common techniques
such as adaptive walks and proportional selection from a population.

\subsection{CGP.jl Library}

\paragraph{}
Cartesian Genetic Programming (CGP) eschews the tree structures more commonly
associated with genetic programming in favor of a grid (hence ``Cartesian'') of
functions that can take as their inputs the outputs from any preceding
functions. Our CGP library\footnote{\url{https://github.com/glesica/CGP.jl}}
attempts to combine computation performance with ease-of-use.

\section{Analysis}

\section{Potential Future Work}

\subsection{Real-Valued Fitness Landscapes}

\paragraph{}
It would be useful to have one or more measures of neutrality that would apply
to real-valued fitness landscapes. The NK landscape is generally understood to
be entirely non-neutral. This is a result of the fact that the fitness of a
given point on the landscape is the sum of randomly chosen real values. In
practice, due to the use of floating point values in place of true real values,
two adjacent points on the landscape might have the same fitness value, but it
is extremely unlikely.

\paragraph{}
However, the biological case for the strict definition of ``neutral'' implied
above is weak. As pointed out by Hughes in~\cite{Hughes2007}:

\begin{displayquote}
An important prediction of the neutral theory is that, when the selection
coefficient in favor of an advantageous mutant or against a deleterious mutant
is less than the reciprocal of twice the effective population size, that mutant
becomes effectively neutral and is not exposed to selection.
\end{displayquote}

\paragraph{}
In the context of an NK landscape, then, neutral neighbors need not have
identical fitness values. The quote above also suggests that there is a
definition of neutrality that might apply comfortably to NK landscapes.

\paragraph{}
Another potential measure of neutrality follows from work by Huynen, Stadler,
and Fontana~\cite{Huynen1996a}. Neutral landscapes exhibit the property that
the genotypic error threshold is lower than the phenotypic error threshold. In
other words, even when the mutation rate is high enough that the population
diffuses away from its initial genotype, the phenotype is still preserved.

\paragraph{}
A significant drawback to this method is that it requires distinct phenotypes,
which the NK model lacks. We may still be able to test it using logic circuits
as a model instead, however.

\paragraph{}
With a definition and measure of neutrality on NK landscapes in-hand, our next
goal will be to determine how changes in the $N$ and $K$ parameters impact
neutrality.

\subsection{Percolation Threshold on NK Landscapes}

\paragraph{}
Gavrilets demonstrated~\cite{Gavrilets1997} that the percolation threshold for
an uncorrelated fitness landscape scales with the inverse of the number of
dimensions under consideration. The percolation threshold is the probability
that a given genotype is part of a particular percolating neutral network
(Gavrilets limited fitness values to ``high'' and ``low'', so the neutral
network in question was the high fitness network). Essentially, the percolation
threshold represents the fraction of the sequence space that must be part of a
neutral network for it to become a percolating network.

\paragraph{}
In other words, as the number of dimensions goes up, it takes a smaller share
of the nodes to create a percolating neutral network. This makes sense because
each additional dimension increases the number of ``routes'' between two
points. This is really quite intuitive for low dimensions. For example, imagine
a video game like Super Mario Brothers in which the character moves in two
dimensions. Many a child has observed in frustration (or perhaps annoyance)
that most of the puzzles in such games would be trivial if only a third
dimension were available to allow greater freedom of movement.

\paragraph{}
We hypothesize that the relationship identified by Gavrilets holds for NK
landscapes as well, using the measure of neutrality we intend to develop.

\subsection{Biologically Realistic Neutral Evolution}

% TODO: Discuss the Fisher -> Kimura path from Orr.

\paragraph{}
While adaptive walks are a popular technique for evolutionary simulations, it
may not be biologically appropriate to simulate a population as a single point
in sequence space.

\paragraph{}
The use of a single point in sequence space to model an entire population
relies on the assumption of strong selective pressure and a low mutation rate.
In this scenario the population will be generally constrained to the dominant
phenotype (see the discussion of error thresholds above).

\paragraph{}
However, in experimental evolution simulations, mutations rates are often quite
high for computational reasons (a higher mutation rate means less computing
power and time are required). Higher rates of mutation also exist in
populations of simpler organisms.

\paragraph{}
In our case, we are also interested in simulating situations with weaker
selection.

\paragraph{}
For the reasons discussed above, we intend to employ more biologically
realistic simulation parameters. We will use proportional selection from a
relatively large population instead of an adaptive walk or other selection
techniques common in computational simulations such as Mu-Lambda.

\bibliographystyle{acm}
\bibliography{fitness}{}
\end{document}
